%%%%%%%%%%%%%%%%%%%%%%%%%%%%%%%%%%%%%%%%%
% Short Sectioned Assignment
% LaTeX Template
% Version 1.0 (5/5/12)
%
% This template has been downloaded from:
% http://www.LaTeXTemplates.com
%
% Original author:
% Frits Wenneker (http://www.howtotex.com)
%
% License:
% CC BY-NC-SA 3.0 (http://creativecommons.org/licenses/by-nc-sa/3.0/)
%
%%%%%%%%%%%%%%%%%%%%%%%%%%%%%%%%%%%%%%%%%

%----------------------------------------------------------------------------------------
%	PACKAGES AND OTHER DOCUMENT CONFIGURATIONS
%----------------------------------------------------------------------------------------

\documentclass[paper=a4, fontsize=11pt]{scrartcl} % A4 paper and 11pt font size

\usepackage[T1]{fontenc} % Use 8-bit encoding that has 256 glyphs
\usepackage{fourier} % Use the Adobe Utopia font for the document - comment this line to return to the LaTeX default
\usepackage[english]{babel} % English language/hyphenation
\usepackage{amsmath,amsfonts,amsthm} % Math packages
\usepackage[utf8]{inputenc}
\usepackage{lipsum} % Used for inserting dummy 'Lorem ipsum' text into the template
\usepackage{xcolor}
\usepackage{graphicx}


\usepackage{listings}
  \usepackage{courier}
 \lstset{
         basicstyle=\footnotesize\ttfamily, % Standardschrift
         %numbers=left,               % Ort der Zeilennummern
         numberstyle=\tiny,          % Stil der Zeilennummern
         %stepnumber=2,               % Abstand zwischen den Zeilennummern
         numbersep=5pt,              % Abstand der Nummern zum Text
         tabsize=2,                  % Groesse von Tabs
         extendedchars=true,         %
         breaklines=true,            % Zeilen werden Umgebrochen
         keywordstyle=\color{red},
    		frame=b,         
 %        keywordstyle=[1]\textbf,    % Stil der Keywords
 %        keywordstyle=[2]\textbf,    %
 %        keywordstyle=[3]\textbf,    %
 %        keywordstyle=[4]\textbf,   \sqrt{\sqrt{}} %
         stringstyle=\color{white}\ttfamily, % Farbe der String
         showspaces=false,           % Leerzeichen anzeigen ?
         showtabs=false,             % Tabs anzeigen ?
         xleftmargin=17pt,
         framexleftmargin=17pt,
         framexrightmargin=5pt,
         framexbottommargin=4pt,
         %backgroundcolor=\color{lightgray},
         showstringspaces=false      % Leerzeichen in Strings anzeigen ?        
 }
 \lstloadlanguages{% Check Dokumentation for further languages ...
         %[Visual]Basic
         %Pascal
         %C
         %C++
         %XML
         %HTML
         Java
 }
    %\DeclareCaptionFont{blue}{\color{blue}} 

  %\captionsetup[lstlisting]{singlelinecheck=false, labelfont={blue}, textfont={blue}}
  \usepackage{caption}
\DeclareCaptionFont{white}{\color{white}}
\DeclareCaptionFormat{listing}{\colorbox[cmyk]{0.43, 0.35, 0.35,0.01}{\parbox{\textwidth}{\hspace{15pt}#1#2#3}}}
\captionsetup[lstlisting]{format=listing,labelfont=white,textfont=white, singlelinecheck=false, margin=0pt, font={bf,footnotesize}}


\usepackage{sectsty} % Allows customizing section commands
\allsectionsfont{\centering \normalfont\scshape} % Make all sections centered, the default font and small caps

\usepackage{fancyhdr} % Custom headers and footers
\pagestyle{fancyplain} % Makes all pages in the document conform to the custom headers and footers
\fancyhead{} % No page header - if you want one, create it in the same way as the footers below
\fancyfoot[L]{} % Empty left footer
\fancyfoot[C]{} % Empty center footer
\fancyfoot[R]{\thepage} % Page numbering for right footer
\renewcommand{\headrulewidth}{0pt} % Remove header underlines
\renewcommand{\footrulewidth}{0pt} % Remove footer underlines
\setlength{\headheight}{13.6pt} % Customize the height of the header

\numberwithin{equation}{section} % Number equations within sections (i.e. 1.1, 1.2, 2.1, 2.2 instead of 1, 2, 3, 4)
\numberwithin{figure}{section} % Number figures within sections (i.e. 1.1, 1.2, 2.1, 2.2 instead of 1, 2, 3, 4)
\numberwithin{table}{section} % Number tables within sections (i.e. 1.1, 1.2, 2.1, 2.2 instead of 1, 2, 3, 4)

\setlength\parindent{0pt} % Removes all indentation from paragraphs - comment this line for an assignment with lots of text

%----------------------------------------------------------------------------------------
%	TITLE SECTION
%----------------------------------------------------------------------------------------

\newcommand{\horrule}[1]{\rule{\linewidth}{#1}} % Create horizontal rule command with 1 argument of height

\title{	
\normalfont \normalsize 
\textsc{ENSEIRB-MATMECA} \\ [25pt] % Your university, school and/or department name(s)
\horrule{0.5pt} \\[0.4cm] % Thin top horizontal rule
\huge N-Bodies : Attraction gravitationnelle. \\ % The assignment title
\horrule{2pt} \\[0.5cm] % Thick bottom horizontal rule
}

\author{Caneill Pierre-Yves, Lux Benjamin} % Your name

\date{\normalsize\today} % Today's date or a custom date

\begin{document}

\maketitle % Print the title

%----------------------------------------------------------------------------------------
%	Version 1
%----------------------------------------------------------------------------------------

\section{Programme MPI pour la simulation d'unn problème à n corps}



\subsection{Principe Gloabal}
% equations / principe phisique / simulation discontinue (=discrete)
% pas assez de mem sur un seul processus

\subsection{Utilisation des type MPI}

\subsection{Anneau de communication}
% on communique avec chacun des autres processus tour par tour / double buffering
A chaque itération, on doit connaitre les positions et les masses des
corps des autres processus afin de calculer la somme des intéractions
sur les corps présent dans le notre.

Pour cela on créer une boucle de communication. Après avoir fait la
somme partielle des forces intèrne au processus on obtient du
processus précédent ses données et on donne les notres au processus
suivant. Une fois la somme partielle complétée, on revoie ces données
au processus suivant et on en récupère des nouvelles, ainsi dessuite.

Au premier tour, le processus $0$ reçoit donc les données de $p-1$, au
tour deux les données de $p-2$.



\subsection{Recouvrement calcul communication}
% Communications non-bloquantes 
Pour ne pas perdre de temps avec l'attentes de communications, qui
sont globalement plus longues que les étapes de calcul, on essaie
toujours avoir des communications de cours.

Pour celà on utlise des communications non-bloquantes. Cependant cette
optimisation est la plus part du temps supérflue. En effet mpi va
généralement lancer ces communications lors de l'appel à
$MPI\_Wait$. Cependant il peut exister des architecture ou mpi va
réellement effectuer ces communications en asynchrone.


\subsection{Factorisation des communications}
% Utilisation de MPI_Send_init / MPI_Recv_init
Puisque l'on envoie nos donnée toujours au même processus, le suivant,
et on ne les reçoit que du précédent, on peut optimiser le code. On
utilise des les fonctions $MPI\_Send\_init$ et $MPI\_Recv\_init$ pour
factoriser ses étapes et préparer à l'avance nos requettes, de type
$MPI\_Request$.\\ On utlise $MPI\_Start$ pour démarrer une requette
ainsi créée. On effectue ensuite nos calculs et enfin on attends la
fin des communications grace à $MPI\_Wait$, pour pouvoir passer à
l'itération suivante.


\subsection{Choix du dt}
% Pour pallier à la non continuitée (?)/  ????

\subsection{Optimisation supplémentaire}
% envisageable ? 
% on calcul toute les forces en double car Fab = -Fba 
% mais si on les calcul qu'une fois, les communications seront elles inférieures au cout de calcul ?


\subsection{Affichage graphique}
% python / un screen-shot ?

\subsection{Perfomances}


\section{Exécuter le programme}

\subsection{Version 1}
\begin{lstlisting}[caption={Exécution de la v1}]
  Mode release : 
     make
  Mode debug :
     make DEBUG=yes

  ./deploy.sh
  ./master n+1 sizeinterval password    (avec n le nombre de slave)

\end{lstlisting}


\section{Conclusion}

La version 1 marche correctement et les résultats mesurés
correspondent aux attentes. Bien que la version 2 s'éxécute
correctement sur nos machines, nous n'arrivons pas à la lancer sur
latifundio. Nous n'avons pas réussis à débuguer cette dernière version
dans le temps imparti. Peut être que des tests plus réguliers de
notre code sur latifundio nous aurait permis de trouver le problème
dès son apparition. 

\end{document}
